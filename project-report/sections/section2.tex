\section{Procura de notícias}
O utilizador, pode fazer procuras, fornecendo uma \textit{query}. Esta é processada pelo \textit{whoosh} que trata de encontrar correspondências devolvendo o \textit{link} da notícia encontrada. Este \textit{link} é depois usado para fazer \textit{queries} à base de dados para ser devolvido os diversos elementos do artigo. Após a procura é devolvido ao utilizador os títulos das notícias, por ordem de relevância, onde a \textit{query} foi encontrada. À frente de cada título, é apresentado as entidades encontradas na noticía. Este último tópico é explicado na secção seguinte.

Os módulos onde esta parte do projecto foi desenvolvida encontram-se na pasta \textit{storageTools}. No ficheiro \textit{whoosh\_tools} estão presentes as funções de indexação e procura. No ficheiro \textit{mongo\_tools} encontram-se as funções para fazer operações na base de dados.
Na secção Anexos, encontram-se vários exemplos de pesquisa de notícias.