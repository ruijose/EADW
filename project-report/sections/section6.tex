\section{Análise Crítica}
Com base nos exemplo que apresentamos nos anexos, verificamos que o nosso projecto podia sofrer bastantes melhorias na parte da recolha de entidades. Observa-se por exemplo, que na procura pela \textit{query} `morte` não foram encontradas quaisquer entidades nas quatro notícias devolvidas. Isto deve-se principalmente ao facto de a ferramenta \textit{nltk} estar preparada para a análise de texto na língua inglesa. Uma melhoria a fazer no futuro seria usar o módulo \textit{floresta} do \textit{nltk} para tentar obter melhores resultados na língua portuguesa. Contudo, com base nos gráficos apresentados na secção estatísticas achamos que os resultados são satisfatórios pois, mesmo assim são encontrados várias entidades portuguesas.

\section{Conclusão}
Após a realização deste projecto, concluímos que podíamos ter obtido melhores resultados em certas partes do projecto. A linguagem \textit{python} é adequada a este tipo de projectos pois tem disponível um grande conjunto de ferramentas que torna a extração e o processamento de informação um processo mais simples.